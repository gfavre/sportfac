%% Generated by Sphinx.
\def\sphinxdocclass{report}
\documentclass[letterpaper,10pt,english]{sphinxmanual}
\ifdefined\pdfpxdimen
   \let\sphinxpxdimen\pdfpxdimen\else\newdimen\sphinxpxdimen
\fi \sphinxpxdimen=.75bp\relax

\usepackage[utf8]{inputenc}
\ifdefined\DeclareUnicodeCharacter
 \ifdefined\DeclareUnicodeCharacterAsOptional\else
  \DeclareUnicodeCharacter{00A0}{\nobreakspace}
\fi\fi
\usepackage{cmap}
\usepackage[T1]{fontenc}
\usepackage{amsmath,amssymb,amstext}
\usepackage{babel}
\usepackage{times}
\usepackage[Bjarne]{fncychap}
\usepackage[dontkeepoldnames]{sphinx}

\usepackage{geometry}

% Include hyperref last.
\usepackage{hyperref}
% Fix anchor placement for figures with captions.
\usepackage{hypcap}% it must be loaded after hyperref.
% Set up styles of URL: it should be placed after hyperref.
\urlstyle{same}

\addto\captionsenglish{\renewcommand{\figurename}{Fig.}}
\addto\captionsenglish{\renewcommand{\tablename}{Table}}
\addto\captionsenglish{\renewcommand{\literalblockname}{Listing}}

\addto\extrasenglish{\def\pageautorefname{page}}

\setcounter{tocdepth}{1}



\title{sportfac_project Documentation}
\date{Jun 23, 2017}
\release{0.1}
\author{ChangeToMyName}
\newcommand{\sphinxlogo}{\vbox{}}
\renewcommand{\releasename}{Release}
\makeindex

\begin{document}

\maketitle
\sphinxtableofcontents
\phantomsection\label{\detokenize{index::doc}}


Contents:


\chapter{Configuration}
\label{\detokenize{configurer:welcome-to-sportfac-project-s-documentation}}\label{\detokenize{configurer:configuration}}\label{\detokenize{configurer::doc}}
\noindent\sphinxincludegraphics{{configure}.png}


\section{Adresse}
\label{\detokenize{configurer:adresse}}
L’adresse du site est celle qui sera communiquée aux utilisateurs, sous la forme, par exemple, \sphinxurl{https://nyon.kepchup.ch}.
Les adresses du type votrenom.ch sont à réserver auprès d’un registrar, par exemple \sphinxhref{http://gandi.net}{Gandi}.

Les adresses du type votrenom.kepchup.ch sont gratuites.
\begin{itemize}
\item {} 
domaine personnel: \_\_\_\_\_\_\_\_\_\_\_\_\_\_\_\_\_\_\_\_\_

\end{itemize}

ou
\begin{itemize}
\item {} 
sous domaine de kepchup: \_\_\_\_\_\_\_\_\_\_\_\_\_\_\_\_\_.kepchup.ch

\end{itemize}


\section{Options}
\label{\detokenize{configurer:options}}
Certaines fonctionnalités peuvent être activées ou désactivées:
\begin{itemize}
\item {} 
Gestion de la facturation (inutile si toutes les activités sont gratuites) oui / non

\item {} 
Gestion des absences oui / non

\end{itemize}


\section{Personnalisation du site}
\label{\detokenize{configurer:personnalisation-du-site}}\begin{itemize}
\item {} 
Titre du site: Il apparait dans la barre de titre du navigateur sur chaque page.

\end{itemize}
\begin{quote}\begin{description}
\item[{Exemple}] \leavevmode
\begin{DUlineblock}{0em}
\item[] Inscription au sport facultatif - EP Coppet
\end{DUlineblock}

\end{description}\end{quote}
\begin{itemize}
\item {} 
Pied de page: il est identique sur toutes les pages. On y place généralement le nom et l’adresse de contact, ainsi qu’un logo.

\end{itemize}
\begin{quote}\begin{description}
\item[{Exemple}] \leavevmode
\begin{DUlineblock}{0em}
\item[] © Établissement primaire de Coppet et environs
\item[] Chemin du Chaucey 7, 1296 Coppet - Tél: +41 22 557 58 58
\item[] 
\item[] {[}logo{]}
\end{DUlineblock}

\end{description}\end{quote}
\begin{itemize}
\item {} 
Logo: utilisé dans le pied de page, en petit dans la barre d’adresse.

\item {} 
Image de fond: image éventuellement utilisée en fond de page (ex: \sphinxurl{https://ssfmontreux.ch} )

\end{itemize}


\subsection{Page d’accueil}
\label{\detokenize{configurer:page-d-accueil}}
Texte de bienvenue, visible sur la page d’accueil.
\begin{quote}\begin{description}
\item[{Exemple}] \leavevmode
\begin{DUlineblock}{0em}
\item[] Le programme du sport scolaire facultatif s’adresse aux élèves de l’Etablissement primaire de Coppet.
\item[] 
\item[] Il propose une série d’activités en rapport avec le sport et se fixe pour objectifs :
\item[]
\begin{DUlineblock}{\DUlineblockindent}
\item[] * de faire découvrir de nouvelles disciplines sportives
\item[] * d’encourager une pratique physique régulière
\item[] * de développer des valeurs éducatives fondamentales
\end{DUlineblock}
\item[] En règle générale, les élèves choisissent des activités nouvelles qu’ils/qu’elles ne pratiquent pas habituellement, dans le but, en cas d’intérêt marqué, de rejoindre les rangs d’un club local.
\item[] 
\item[] \sphinxstylestrong{En participant, chaque enfant s’engage à respecter les conditions de participation et, par son attitude, à faire en sorte que l’activité soit bénéfique tant pour lui-même que pour l’ensemble de ses camarades.}
\item[] 
\item[] \sphinxstyleemphasis{Nous souhaitons à toutes et tous beaucoup de plaisir dans la pratique des activités de ce programme que nous espérons varié et attrayant.}
\end{DUlineblock}

\end{description}\end{quote}


\subsection{Page de contact}
\label{\detokenize{configurer:page-de-contact}}
À droite du formulaire de contact se trouvent les coordonnées complètes. \sphinxurl{https://coppet.kepchup.ch/contact/}
\begin{quote}\begin{description}
\item[{Exemple}] \leavevmode
\begin{DUlineblock}{0em}
\item[] \sphinxstylestrong{Établissement Primaire de Coppet et Environs}
\item[] 
\item[] Chemin du Chaucey 7
\item[] CH-1296 Coppet
\item[] \sphinxstylestrong{Téléphone}  +41 22 557 58 58
\item[] \sphinxstylestrong{Fax} +41 22 557 58 59
\item[] \sphinxstylestrong{Email} \sphinxhref{mailto:ep.coppet@vd.ch}{ep.coppet@vd.ch}
\item[] 
\item[] 
\item[] \sphinxstylestrong{Responsable Sport Scolaire Facultatif}
\item[] 
\item[] Remo Aeschbach
\item[] Chemin du Chaucey 7
\item[] CH-1296 Coppet
\item[] \sphinxstylestrong{Téléphone} +41 22 557 58 58
\item[] \sphinxstylestrong{Portable} +41 79 417 69 93
\item[] \sphinxstylestrong{Email} \sphinxhref{mailto:remo.aeschbach@vd.educanet2.ch}{remo.aeschbach@vd.educanet2.ch}
\end{DUlineblock}

\end{description}\end{quote}


\subsection{Page du règlement de participation}
\label{\detokenize{configurer:page-du-reglement-de-participation}}
Un règlement de participation est à prévoir. À titre d’exemple, celui de Coppet: \sphinxurl{https://coppet.kepchup.ch/reglement/}


\subsection{Page des paiements}
\label{\detokenize{configurer:page-des-paiements}}
Les utilisateurs paient par virement bancaire, ou en apportant l’argent directement. Il n’y a pas de système de paiement en ligne pour le moment.
\begin{itemize}
\item {} 
Texte d’explication.

\end{itemize}
\begin{quote}\begin{description}
\item[{Exemple}] \leavevmode
\begin{DUlineblock}{0em}
\item[] Nous vous remercions de verser cette somme d’ici au 31.10.2019 en privilégiant le virement bancaire sur le compte :
\item[] IBAN: CH77 0076 7000 C507 0682 4
\item[] Adresse: AIC, 1201 Genève
\end{DUlineblock}

\end{description}\end{quote}
\begin{itemize}
\item {} 
Délai de paiement en jours depuis la fin des inscriptions (ex: 30)

\end{itemize}


\section{Inscriptions}
\label{\detokenize{configurer:inscriptions}}\begin{itemize}
\item {} 
Nombre maximum d’inscriptions par élève (ex: 4)

\end{itemize}


\section{Emails}
\label{\detokenize{configurer:emails}}
Beaucoup d’emails peuvent être envoyés par le système: rappel d’inscription, de paiement, liste des participants d’un cours, etc.
\begin{itemize}
\item {} 
Adresse utilisée pour l’envoi des emails automatiques

\item {} 
Signature au bas de chaque email automatique, par exemple:

\end{itemize}
\begin{quote}\begin{description}
\item[{Exemple}] \leavevmode
\begin{DUlineblock}{0em}
\item[] Remo Aeschbach
\item[] Doyen - responsable du sport scolaire facultatif
\item[] EPCoppet
\item[] Chemin du Chaucey 7
\item[] 1296 Coppet
\item[] \sphinxhref{mailto:remo.aeschbach@vd.educanet2.ch}{remo.aeschbach@vd.educanet2.ch}
\item[] +4122 \textbar{} 557 58 58
\item[] +4179 \textbar{} 417 69 93
\end{DUlineblock}

\end{description}\end{quote}


\subsection{Fin des inscriptions}
\label{\detokenize{configurer:fin-des-inscriptions}}
Email envoyé aux utilisateurs qui ont commencé leur inscription, mais ne l’ont pas terminée.
\begin{quote}\begin{description}
\item[{Sujet}] \leavevmode
Votre inscription au sport scolaire facultatif​​​​

\item[{Message}] \leavevmode
\begin{DUlineblock}{0em}
\item[] Madame, Monsieur,
\item[] 
\item[] En passant en revue les inscriptions aux sports scolaires facultatifs, nous constatons que les inscriptions pour votre/vos enfant/s ne sont à ce jour pas encore confirmées (passage à l’étape du paiement).
\item[] 
\item[] Nous vous serions reconnaissants de bien vouloir contrôler les inscriptions que vous avez saisies, de les modifier si nécessaire et de confirmer d’ici à demain soir afin que nous puissions terminer le processus d’inscription.
\item[] 
\item[] Le site des inscriptions: \sphinxurl{https://votrenom.com}
\item[] 
\item[] Enfin, nous vous saurions gré de verser le montant relatif à ces inscriptions sur le compte indiqué.
\item[] En vous remerciant de votre collaboration, nous vous adressons nos cordiaux messages.
\end{DUlineblock}

\end{description}\end{quote}


\subsection{​Rappel de paiement}
\label{\detokenize{configurer:rappel-de-paiement}}
Envoyé aux utilisateurs qui n’ont pas encore payé. Le montant dû est calculé en fonction de l’utilisateur.
\begin{quote}\begin{description}
\item[{Sujet}] \leavevmode
Votre inscription au sport scolaire facultatif​​​​ - Rappel

\item[{Message}] \leavevmode
\begin{DUlineblock}{0em}
\item[] Madame, Monsieur,
\item[] 
\item[] À ce jour jour et sauf erreur de notre part, nous n’avons pas reçu votre paiement pour les activités de sport scolaire facultatif de votre enfant.
\item[] Nous vous saurions gré d’effectuer votre versement:
\item[] 
\item[] total dû: CHF xxx.-
\item[] 
\item[] sur le compte :
\item[] IBAN: CH77 0076 7000 C507 0682 4
\item[] Adresse: AIIP, 1201 Genève
\item[] 
\item[] en précisant votre identifiant dans les communications: xxx
\item[] 
\item[] Vous pouvez également passer à notre secrétariat (avec une copie imprimée du présent mail) qui pourra encaisser directement votre finance d’inscription.
\item[] En vous remerciant d’ores et déjà de votre prompte réaction, nous vous adressons nos cordiaux messages.
\end{DUlineblock}

\end{description}\end{quote}


\subsection{Infos pour le moniteur}
\label{\detokenize{configurer:infos-pour-le-moniteur}}
Les moniteurs des cours reçoivent avant le début de leur cours un email personnalisé contenant divers formulaires (déclaration d’heures pour le canton, liste des participants, feuilles de présence).
\begin{quote}\begin{description}
\item[{Sujet}] \leavevmode
Sport scolaire facultatif - documents monitrice/moniteur cours: {[}N° cours{]} - {[}nom du cours​{]}

\item[{Message}] \leavevmode
\begin{DUlineblock}{0em}
\item[] Chère monitrice, cher moniteur,
\item[] 
\item[] Nous te remercions de t’engager dans l’animation d’une activité du sport scolaire facultatif primaire au sein de notre région et ainsi contribuer à promouvoir la pratique sportive auprès de nos élèves.
\item[] Tu trouveras en pièce jointe tous les documents relatifs au cours dont tu as la charge et qui débute prochainement :
\item[] 
\item[] • liste de tous les cours organisés, avec les informations détaillées de lieux, dates et heures
\item[] • liste des participants avec les n° en cas d’urgence
\item[] • liste de présence à retourner dès la fin du cours
\item[] • feuille de décompte monitrice/moniteur, à retourner dès la fin du cours également
\item[] 
\item[] Tu as la responsabilité de prendre toutes les mesures nécessaires lors d’une absence éventuelle à l’une ou l’autre de tes leçons, soit :
\item[] 
\item[] • dans toute la mesure du possible, te faire remplacer par une personne compétente
\item[] • dans l’impossibilité de te faire remplacer, prévenir tous les participants afin d’éviter le déplacement inutile de ces derniers sur le lieu du cours
\item[] • communiquer ton absence au secrétariat primaire (022-557.58.58)
\item[] 
\item[] Afin que nous puissions te régler dans les meilleurs délais, nous te prions de nous retourner, dès la fin d’un cours, mais au plus tard à la fin de l’année scolaire, les documents suivants, dûment complétés :
\item[] 
\item[] • feuille de décompte, à compléter à l’écran, imprimer et signer - 1 feuille par cours et par monitrice/moniteur
\item[] • liste de présence
\item[] 
\item[] Nous restons à ta disposition pour tout complément d’information, te souhaitons bonne réception de ce courriel et plein succès dans ces activités.
\item[] 
\item[] Meilleurs messages,
\end{DUlineblock}

\end{description}\end{quote}


\subsection{Début d’un cours}
\label{\detokenize{configurer:debut-d-un-cours}}
Les parents sont informés par un mail personnalisé pour chacune de leur inscriptions.
\begin{quote}\begin{description}
\item[{Sujet}] \leavevmode
Sport scolaire facultatif - documents monitrice/moniteur cours: {[}N° cours{]} - {[}nom du cours​{]}

\item[{Message}] \leavevmode
\begin{DUlineblock}{0em}
\item[] Madame, Monsieur,
\item[] 
\item[] Nous avons le plaisir d’inviter votre enfant {[}Prénom et nom de l’enfant{]} à la première séance du cours suivant :
\item[] 
\item[] {[}nom du cours{]}
\item[] Responsable : {[}nom du moniteur{]}
\item[] Jour : {[}jour de la semaine{]} de {[}heure de début{]} à {[}heure de fin{]}
\item[] Date 1ère séance : {[}date{]}
\item[] Nombre de séances : {[}nombre{]}
\item[] Rendez-vous/lieu du cours : {[}lieu{]}
\item[] 
\item[] L’animatrice/l’animateur vous donnera toutes les informations en lien avec son cours lors de la 1ère séance.
\item[] 
\item[] En restant à votre disposition pour tout complément d’information, nous vous adressons, Madame, Monsieur, nos cordiaux messages.
\end{DUlineblock}

\end{description}\end{quote}


\subsection{Notification d’absence}
\label{\detokenize{configurer:notification-d-absence}}
Si la fonctionnalité de gestion des absences est utilisée, nous pouvons configurer l’envoi automatique d’un email aux
parents après une absence.
\begin{itemize}
\item {} 
Délai entre le cours et la notification en jours (ex: 1)

\end{itemize}
\begin{quote}\begin{description}
\item[{Sujet}] \leavevmode
Sport scolaire facultatif - Absence de {[}prénom de l’élève{]} lors du cours {[}nom du cours{]}

\item[{Message}] \leavevmode
\begin{DUlineblock}{0em}
\item[] Madame, Monsieur,
\item[] 
\item[] Votre enfant, {[}prénom nom{]}, était absent lors du cours {[}nom du cours{]} du {[}date{]} :
\item[] 
\item[] Nous vous remercions, à l’avenir de transmettre une excuse par SMS au moniteur:
\item[] * {[}nom instructeur{]}: {[}numéro de téléphone{]}
\item[] 
\item[] En restant à votre disposition pour tout complément d’information, nous vous adressons, Madame, Monsieur, nos cordiaux messages.
\end{DUlineblock}

\end{description}\end{quote}


\section{Admininistrateur}
\label{\detokenize{configurer:admininistrateur}}
Une personne au moins devra être administrateur du site.
\begin{itemize}
\item {} 
Prénom

\item {} 
Nom

\item {} 
email

\end{itemize}


\chapter{Install}
\label{\detokenize{install::doc}}\label{\detokenize{install:install}}
This is where you write how to get a new laptop to run this project.

\begin{sphinxVerbatim}[commandchars=\\\{\}]
\PYG{n+nb}{export} \PYG{n+nv}{PROJECT}\PYG{o}{=}nyon

\PYG{n+nb}{export} \PYG{n+nv}{BASE\PYGZus{}DIR}\PYG{o}{=}\PYGZti{}/Documents/Projets
\PYG{n+nb}{export} \PYG{n+nv}{PROJECT\PYGZus{}NAME}\PYG{o}{=}sportfac\PYGZus{}\PYG{n+nv}{\PYGZdl{}PROJECT}
\PYG{n+nb}{export} \PYG{n+nv}{PROJECT\PYGZus{}DIR}\PYG{o}{=}\PYG{n+nv}{\PYGZdl{}BASE\PYGZus{}DIR}/\PYG{n+nv}{\PYGZdl{}PROJECT\PYGZus{}NAME}
\PYG{n+nb}{export} \PYG{n+nv}{DB\PYGZus{}NAME}\PYG{o}{=}\PYG{n+nv}{\PYGZdl{}PROJECT\PYGZus{}NAME}
\PYG{n+nb}{export} \PYG{n+nv}{DB\PYGZus{}USER}\PYG{o}{=}\PYG{n+nv}{\PYGZdl{}PROJECT\PYGZus{}NAME}
\PYG{n+nb}{export} \PYG{n+nv}{DB\PYGZus{}PASSWORD}\PYG{o}{=}\PYG{n+nv}{\PYGZdl{}PROJECT\PYGZus{}NAME}
\PYG{n+nb}{export} \PYG{n+nv}{VENVDIR}\PYG{o}{=}\PYGZti{}/.virtualenvs/\PYG{n+nv}{\PYGZdl{}PROJECT\PYGZus{}NAME}

git clone http://git.pygreg.ch/sportfac.git \PYG{n+nv}{\PYGZdl{}PROJECT\PYGZus{}DIR}
\PYG{n+nb}{cd} \PYG{n+nv}{\PYGZdl{}PROJECT\PYGZus{}DIR}
git checkout \PYGZhy{}b \PYG{n+nv}{\PYGZdl{}PROJECT}
git push \PYGZhy{}\PYGZhy{}set\PYGZhy{}upstream origin \PYG{n+nv}{\PYGZdl{}PROJECT}
mkvirtualenv \PYG{n+nv}{\PYGZdl{}PROJECT\PYGZus{}NAME}
\PYG{n+nb}{echo} \PYG{l+s+sb}{{}`}\PYG{n+nb}{pwd}\PYG{l+s+sb}{{}`} \PYGZgt{} \PYG{n+nv}{\PYGZdl{}VENVDIR}/.project

\PYG{c+c1}{\PYGZsh{} Database}
\PYG{n+nb}{echo} \PYG{l+s+s2}{\PYGZdq{}}\PYG{l+s+s2}{CREATE ROLE }\PYG{n+nv}{\PYGZdl{}DB\PYGZus{}USER}\PYG{l+s+s2}{ WITH LOGIN UNENCRYPTED PASSWORD \PYGZsq{}}\PYG{n+nv}{\PYGZdl{}DB\PYGZus{}PASSWORD}\PYG{l+s+s2}{\PYGZsq{}}\PYG{l+s+s2}{\PYGZdq{}} \PYG{p}{\textbar{}} psql \PYGZhy{}\PYGZhy{}user postgres
\PYG{n+nb}{echo} \PYG{l+s+s2}{\PYGZdq{}}\PYG{l+s+s2}{CREATE DATABASE }\PYG{n+nv}{\PYGZdl{}DB\PYGZus{}NAME}\PYG{l+s+s2}{ WITH OWNER=}\PYG{n+nv}{\PYGZdl{}DB\PYGZus{}USER}\PYG{l+s+s2}{\PYGZdq{}} \PYG{p}{\textbar{}} psql \PYGZhy{}\PYGZhy{}user postgres

\PYG{c+c1}{\PYGZsh{} Env vars}
\PYG{n+nb}{echo} \PYG{l+s+s2}{\PYGZdq{}}\PYG{l+s+s2}{export PYTHONPATH=}\PYG{n+nv}{\PYGZdl{}PROJECT\PYGZus{}DIR}\PYG{l+s+s2}{/sportfac}\PYG{l+s+s2}{\PYGZdq{}} \PYGZgt{}\PYGZgt{} \PYG{n+nv}{\PYGZdl{}VENVDIR}/bin/postactivate
\PYG{n+nb}{echo} \PYG{l+s+s1}{\PYGZsq{}export DJANGO\PYGZus{}SETTINGS\PYGZus{}MODULE=\PYGZdq{}sportfac.settings.local\PYGZdq{}\PYGZsq{}} \PYGZgt{}\PYGZgt{} \PYG{n+nv}{\PYGZdl{}VENVDIR}/bin/postactivate
\PYG{n+nb}{echo} \PYG{l+s+s2}{\PYGZdq{}}\PYG{l+s+s2}{export DB\PYGZus{}USER=}\PYG{n+nv}{\PYGZdl{}DB\PYGZus{}USER}\PYG{l+s+s2}{\PYGZdq{}} \PYGZgt{}\PYGZgt{} \PYG{n+nv}{\PYGZdl{}VENVDIR}/bin/postactivate
\PYG{n+nb}{echo} \PYG{l+s+s2}{\PYGZdq{}}\PYG{l+s+s2}{export DB\PYGZus{}PASSWORD=}\PYG{n+nv}{\PYGZdl{}DB\PYGZus{}PASSWORD}\PYG{l+s+s2}{\PYGZdq{}} \PYGZgt{}\PYGZgt{} \PYG{n+nv}{\PYGZdl{}VENVDIR}/bin/postactivate
\PYG{n+nb}{echo} \PYG{l+s+s2}{\PYGZdq{}}\PYG{l+s+s2}{export DB\PYGZus{}NAME=}\PYG{n+nv}{\PYGZdl{}DB\PYGZus{}NAME}\PYG{l+s+s2}{\PYGZdq{}} \PYGZgt{}\PYGZgt{} \PYG{n+nv}{\PYGZdl{}VENVDIR}/bin/postactivate
\PYG{n+nb}{echo} \PYG{l+s+s2}{\PYGZdq{}export SECRET\PYGZus{}KEY=gdhsagkdahjsg\PYGZdq{}} \PYGZgt{}\PYGZgt{} \PYG{n+nv}{\PYGZdl{}VENVDIR}/bin/postactivate
\PYG{n+nb}{echo} \PYG{l+s+s2}{\PYGZdq{}export PHANTOMJS=/usr/local/bin/phantomjs\PYGZdq{}} \PYGZgt{}\PYGZgt{} \PYG{n+nv}{\PYGZdl{}VENVDIR}/bin/postactivate


\PYG{c+c1}{\PYGZsh{} soft}
pip install \PYGZhy{}r requirements/local.txt
django\PYGZhy{}admin syncdb
django\PYGZhy{}admin migrate
django\PYGZhy{}admin loaddata sportfac/sportfac/fixtures/flatpages.json
\end{sphinxVerbatim}


\chapter{Indices and tables}
\label{\detokenize{index:indices-and-tables}}\begin{itemize}
\item {} 
\DUrole{xref,std,std-ref}{genindex}

\item {} 
\DUrole{xref,std,std-ref}{modindex}

\item {} 
\DUrole{xref,std,std-ref}{search}

\end{itemize}



\renewcommand{\indexname}{Index}
\printindex
\end{document}